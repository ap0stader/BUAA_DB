\subsection{数据库关系模式定义}
\subsubsection{登录信息}
\textsf{登录信息:}(\uline{身份标识符}, 密码密文, 用户角色)
\textsl{注:用户角色共分为学生、教师、学院教务、学校管理员。类型为枚举(8),枚举的实际值为 0-3。}

\subsubsection{用户信息}
\textsf{学生信息:}(\uline{学号}, 姓名, 性别, 联系方式, 班级号) 
\textsl{注:学号是外键,引用登录信息的身份标识符,并且该身份标识符对应的登录信息的用户角色应为学生。班级号是外键,引用班级信息的班级号。}

\textsf{教师信息:}(\uline{工号}, 姓名, 性别, 联系方式, 院系号)
\textsl{注:工号是外键,引用登录信息的身份标识符,并且该身份标识符对应的登录信息的用户角色应为教师。院系号是外键,引用院系信息的院系号。}

\textsf{学院教务信息:}(\uline{工号}, 姓名, 性别, 联系方式, 院系号)
\textsl{注:工号是外键,引用登录信息的身份标识符,并且该身份标识符对应的登录信息的用户角色应为学院教务。院系号是外键,引用院系信息的院系号。}

\subsubsection{组织信息}
\textsf{院系信息:}(\uline{院系号}, 院系名称)

\textsf{专业信息:}(\uline{专业号}, 专业名称, 院系号)
\textsl{注:院系号是外键,引用院系信息的院系号。}

\textsf{班级信息:}(\uline{班级号}, 专业号, 班主任工号)
\textsl{注:专业号是外键,引用专业信息的专业号。班主任工号是外键,引用教师信息的工号。}

\subsubsection{课程信息}
\textsf{课程信息:}(\uline{课程编号}, 课程名称, 课程分类, 学分, 学时,申报教师工号)
\textsl{注:申报教师工号是外键,引用教师信息的工号。}

\textsf{学期信息:}(\uline{学期编号}, 学期名称)

\textsf{教学班信息:}(\uline{教学班编号}, 课程编号, 教师工号, 对内容量, 对外容量, 额外说明, 开设学期)
\textsl{注:课程编号是外键,引用课程信息的课程编号。教师工号是外键,引用教师信息的工号。开设学期是外键,引用学期信息的学期编号。}

\textsf{场地信息:}(\uline{场地编号}, 场地名称)

\textsf{场地资源信息:}(\uline{场地资源编号}, 场地编号, 可用时间)
\textsl{注:场地编号是外键,引用场地信息的场地编号。}

\textsf{场地资源使用信息:}(\uline{教学班编号, 场地资源编号})
\textsl{注:教学班编号是外键,引用教学班信息的教学班编号。场地资源编号是外键,引用场地资源信息的场地资源编号。}

\subsubsection{选课信息}
\textsf{选课信息:}(\uline{学号, 教学班编号}, 成绩, 评教分数)
\textsl{注:学号是外键,引用学生信息的学号。教学班编号是外键,引用教学班信息的教学班编号。}

\textsf{预选信息:}(\uline{学号, 教学班编号}, 志愿顺序, 自我介绍)
\textsl{注:学号是外键,引用学生信息的学号。教学班编号是外键,引用教学班信息的教学班编号。}

\subsubsection{审计信息}
\textsf{登录审计信息:}(\uline{登录审计编号}, 声称身份标识符, 登录时间, 登录结果)

\textsf{选课审计信息:}(\uline{选课审计编号}, 学号, 操作教学班编号, 操作类型, 操作时间, 操作人员身份标识符)
\textsl{注:学号是外键,引用学生信息的学号。操作教学班编号是外键,引用教学班信息的教学班编号。操作人员身份标识符是外键,引用登录信息的身份标识符。}
