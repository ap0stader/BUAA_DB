\subsection{关系模式范式等级的判定与规范化}
\ttfamily
以上所有关系的范式等级均达到了\textnormal{\textbf{BCNF}},均超过了\textnormal{\textbf{3NF}},因此不必要进行进一步的规范化处理。下为对各关系的范式等级的判定过程:

\normalfont

\begin{enumerate}
    \item (\uline{身份标识符}, 密码密文, 用户角色) \par
    其中的函数依赖有: \newline
    $\text{身份标识符} \rightarrow \text{密码密文}$ \newline
    $\text{身份标识符} \rightarrow \text{用户角色}$ \par
    这些函数依赖右侧的属性均不是左侧属性的字集,左侧的属性均为码,因此该关系满足\textbf{BCNF}。

    \item (\uline{学号}, 姓名, 性别, 联系方式, 班级号) \par
    其中的函数依赖有: \newline
    $\text{学号} \rightarrow \text{姓名}$ \newline
    $\text{学号} \rightarrow \text{性别}$ \newline
    $\text{学号} \rightarrow \text{联系方式}$ \newline
    $\text{学号} \rightarrow \text{班级号}$ \par
    这些函数依赖右侧的属性均不是左侧属性的字集,左侧的属性均为码,因此该关系满足\textbf{BCNF}。

    \item (\uline{工号}, 姓名, 性别, 联系方式, 院系号) \par
    其中的函数依赖有: \newline
    $\text{工号} \rightarrow \text{姓名}$ \newline
    $\text{工号} \rightarrow \text{性别}$ \newline
    $\text{工号} \rightarrow \text{联系方式}$ \newline
    $\text{工号} \rightarrow \text{院系号}$ \par
    这些函数依赖右侧的属性均不是左侧属性的字集,左侧的属性均为码,因此该关系满足\textbf{BCNF}。
    
    \item (\uline{工号}, 姓名, 性别, 联系方式, 院系号) \par
    其中的函数依赖有: \newline
    $\text{工号} \rightarrow \text{姓名}$ \newline
    $\text{工号} \rightarrow \text{性别}$ \newline
    $\text{工号} \rightarrow \text{联系方式}$ \newline
    $\text{工号} \rightarrow \text{院系号}$ \par
    这些函数依赖右侧的属性均不是左侧属性的字集,左侧的属性均为码,因此该关系满足\textbf{BCNF}。
    
    \item (\uline{院系号}, 院系名称) \par
    其中的函数依赖有: \newline
    $\text{院系号} \rightarrow \text{院系名称}$ \par
    这些函数依赖右侧的属性均不是左侧属性的字集,左侧的属性均为码,因此该关系满足\textbf{BCNF}。
    
    \item (\uline{专业号}, 专业名称, 院系号) \par
    其中的函数依赖有: \newline
    $\text{专业号} \rightarrow \text{专业名称}$ \newline
    $\text{专业号} \rightarrow \text{院系号}$ \par
    这些函数依赖右侧的属性均不是左侧属性的字集,左侧的属性均为码,因此该关系满足\textbf{BCNF}。
    
    \item (\uline{班级号}, 专业号, 班主任工号) \par
    其中的函数依赖有: \newline
    $\text{班级号} \rightarrow \text{专业号}$ \newline
    $\text{班级号} \rightarrow \text{班主任工号}$ \par
    这些函数依赖右侧的属性均不是左侧属性的字集,左侧的属性均为码,因此该关系满足\textbf{BCNF}。
    
    \item (\uline{课程编号}, 课程名称, 课程分类, 学分, 学时,申报教师工号) \par
    其中的函数依赖有: \newline
    $\text{课程编号} \rightarrow \text{课程名称}$ \newline
    $\text{课程编号} \rightarrow \text{课程分类}$ \newline
    $\text{课程编号} \rightarrow \text{学分}$ \newline
    $\text{课程编号} \rightarrow \text{学时}$ \newline
    $\text{课程编号} \rightarrow \text{申报教师工号}$ \par
    这些函数依赖右侧的属性均不是左侧属性的字集,左侧的属性均为码,因此该关系满足\textbf{BCNF}。
    
    \item (\uline{学期编号}, 学期名称) \par
    其中的函数依赖有: \newline
    $\text{学期编号} \rightarrow \text{学期名称}$ \par
    这些函数依赖右侧的属性均不是左侧属性的字集,左侧的属性均为码,因此该关系满足\textbf{BCNF}。
    
    \item (\uline{教学班编号}, 课程编号, 教师工号, 对内容量, 对外容量, 额外说明, 开设学期) \par
    其中的函数依赖有: \newline
    $\text{教学班编号} \rightarrow \text{课程编号}$ \newline
    $\text{教学班编号} \rightarrow \text{教师工号}$ \newline
    $\text{教学班编号} \rightarrow \text{对内容量}$ \newline
    $\text{教学班编号} \rightarrow \text{对外容量}$ \newline
    $\text{教学班编号} \rightarrow \text{额外说明}$ \newline
    $\text{教学班编号} \rightarrow \text{开设学期}$ \par
    这些函数依赖右侧的属性均不是左侧属性的字集,左侧的属性均为码,因此该关系满足\textbf{BCNF}。
    
    \item (\uline{场地编号}, 场地名称) \par
    其中的函数依赖有: \newline
    $\text{场地编号} \rightarrow \text{场地名称}$ \par
    这些函数依赖右侧的属性均不是左侧属性的字集,左侧的属性均为码,因此该关系满足\textbf{BCNF}。
    
    \item (\uline{场地资源编号}, 场地编号, 可用时间) \par
    其中的函数依赖有: \newline
    $\text{场地资源编号} \rightarrow \text{场地编号}$ \newline
    $\text{场地资源编号} \rightarrow \text{可用时间}$ \par
    这些函数依赖右侧的属性均不是左侧属性的字集,左侧的属性均为码,因此该关系满足\textbf{BCNF}。
    
    \item (\uline{教学班编号, 场地资源编号}) \par
    该关系的唯一的码是全码,因此该关系满足\textbf{BCNF}。
    
    \item (\uline{学号, 教学班编号}, 成绩, 评教分数) \par
    其中的函数依赖有: \newline
    $\text{学号, 教学班编号} \rightarrow \text{成绩}$ \newline
    $\text{学号, 教学班编号} \rightarrow \text{评教分数}$ \par
    这些函数依赖右侧的属性均不是左侧属性的字集,左侧的属性均为码,因此该关系满足\textbf{BCNF}。
    
    \item (\uline{学号, 教学班编号}, 志愿顺序, 自我介绍) \par
    其中的函数依赖有: \newline
    $\text{学号, 教学班编号} \rightarrow \text{志愿顺序}$ \newline
    $\text{学号, 教学班编号} \rightarrow \text{自我介绍}$ \par
    这些函数依赖右侧的属性均不是左侧属性的字集,左侧的属性均为码,因此该关系满足\textbf{BCNF}。
    
    \item (\uline{登录审计编号}, 声称身份标识符, 登录时间, 登录结果) \par
    其中的函数依赖有: \newline
    $\text{审计编号} \rightarrow \text{声称身份标识符}$ \newline
    $\text{审计编号} \rightarrow \text{登录时间}$ \newline
    $\text{审计编号} \rightarrow \text{登录结果}$ \par
    这些函数依赖右侧的属性均不是左侧属性的字集,左侧的属性均为码,因此该关系满足\textbf{BCNF}。
    
    \item (\uline{选课审计编号}, 学号, 操作教学班编号, 操作类型, 操作时间, 操作人员身份标识符) \par
    其中的函数依赖有: \newline
    $\text{选课审计编号} \rightarrow \text{学号}$ \newline
    $\text{选课审计编号} \rightarrow \text{操作教学班编号}$ \newline
    $\text{选课审计编号} \rightarrow \text{操作类型}$ \newline
    $\text{选课审计编号} \rightarrow \text{操作时间}$ \newline
    $\text{选课审计编号} \rightarrow \text{操作人员身份标识符}$ \par
    这些函数依赖右侧的属性均不是左侧属性的字集,左侧的属性均为码,因此该关系满足\textbf{BCNF}。
    
\end{enumerate}
