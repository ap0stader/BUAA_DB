\subsection{数据元素表}

\begin{center}
    \footnotesize
    \begin{longtable}{p{6em}p{16em}p{8em}@{}p{2em}p{16em}}
        \toprule
        \multirow{2}{*}{\textbf{数据项}} &
            \multicolumn{4}{c}{\textbf{特征}} \\
        \cmidrule{2-5}
        & 描述 & 类型(宽度) & 可空 & 完整性约束 \\
        \midrule
        \endhead
        \bottomrule
        \endfoot

        \multicolumn{5}{c}{\textbf{数据组1:学生信息}} \\
        \midrule
        学号 & 用于一一对应标识学生 & 字符串(32) & 否 & 学号不重复 \\
        姓名 & 学生姓名 & 字符串(32) & 否 & 无 \\
        性别 & 学生性别 & 字符串(4) & 是 & “男”或“女” \\
        联系方式 & 学生手机号 & 字符串(16) & 是 & 合法的手机号 \\
        所属班级号 & 学生所在班级的班级号 & 整数(32) & 否 & 必须为某个存在的班级的班级号 \\
        所属专业号 & 学生所在专业的专业号 & 整数(16) & 否 & 必须为某个存在的专业的专业号 \\
        所属院系号 & 学生所在院系的院系号 & 整数(8) & 否 & 必须为某个存在的院系的院系号 \\
        \midrule
        
        \multicolumn{5}{c}{\textbf{数据组2:教师信息}} \\
        \midrule
        工号 & 用于一一对应标识教师 & 字符串(32) & 否 & 工号不重复 \\
        姓名 & 教师姓名 & 字符串(32) & 否 & 无 \\
        性别 & 教师性别 & 字符串(4) & 是 & “男”或“女” \\
        联系方式 & 教师手机号 & 字符串(16) & 是 & 合法的手机号 \\
        所属院系号 & 教师所在院系的院系号 & 整数(8) & 否 & 必须为某个存在的院系的院系号 \\
        班主任班级号 & 该教师担任班主任的班级号 & 整数(32) & 是 & 必须为某个存在的班级的班级号 \\
        \midrule
        
        \multicolumn{5}{c}{\textbf{数据组3:学院教务信息}} \\
        \midrule
        工号 & 用于一一对应标识学院教务 & 字符串(32) & 否 & 工号不重复 \\   
        姓名 & 教务姓名 & 字符串(32) & 否 & 无 \\
        性别 & 教务性别 & 字符串(4) & 是 & “男”或“女” \\
        联系方式 & 教务手机号 & 字符串(16) & 是 & 合法的手机号 \\
        所属院系号 & 教务所在院系的院系号 & 整数(8) & 否 & 必须为某个存在的院系的院系号 \\
        \midrule

        \multicolumn{5}{c}{\textbf{数据组4:院系信息}} \\
        \midrule
        院系号 & 用于一一对应标识学院 & 整数(8) & 否 & 院系号不重复 \\
        院系名称 & 学院名称 & 字符串(32) & 否 & 无 \\
        \midrule

        \multicolumn{5}{c}{\textbf{数据组5:专业信息}} \\
        \midrule
        专业号 & 用于一一对应标识专业 & 整数(16) & 否 & 专业号\textbf{在所有学院间}不重复 \\
        专业名称 & 专业名称 & 字符串(32) & 否 & 无 \\
        所属院系号 & 专业所属院系的院系号 & 整数(8) & 否 & 必须为某个存在的院系的院系号 \\
        \midrule

        \multicolumn{5}{c}{\textbf{数据组6:班级信息}} \\
        \midrule
        班级号 & 用于一一对应标识班级 & 整数(32) & 否 & 班级号\textbf{在所有专业间}不重复 \\
        所属院系号 & 班级所属院系的院系号 & 整数(8) & 否 & 必须为某个存在的院系的院系号 \\
        所属专业号 & 班级所属专业的专业号 & 整数(16) & 否 & 必须为某个存在的专业的专业号 \\
        班主任工号 & 班级的班主任的工号。允许暂时没有班主任 & 字符串(32) & 是 & 必须为某个存在的教师的工号 \\
        \midrule

        \multicolumn{5}{c}{\textbf{数据组7:课程信息}} \\
        \midrule
        课程编号 & 某一个课程的长期固定编号,一个课程可以有多个教师于多个学期开课 & 字符串(24) & 否 & 课程编号不重复 \\
        课程名称 & 课程名称 & 字符串(32) & 否 & 无 \\
        课程分类 & 课程分类分类,共有必修课、选修课、通识课、体育课、科研课五类 & 枚举(8) & 否 & 枚举的实际值为0-4 \\
        学分 & 课程学分 & 浮点数(32) & 否 & 0.1至12.0之间的0.1的倍数 \\
        学时 & 课程学时 & 整数(16) & 否 & 0至640之间的整数 \\
        申报教师工号 & 申报该课程的教师工号 & 字符串(32) & 否 & 必须为某个存在的教师的工号 \\   
        \midrule

        \multicolumn{5}{c}{\textbf{数据组8:场地资源信息}} \\
        \midrule
        场地资源编号 & 用于一一对应标识场地资源,一个场地资源对应一个学时 & 整数(32) & 否 & 场地资源编号不重复。第31-16位表示场地,第15-0位表示时间 \\
        场地名称 & 场地名称 & 字符串(32) & 否 & 无 \\
        可用时间 & 该场地可用于教学活动的时间 & 整数(16) & 否 & 第15-8位表示周次,第7-4位表示星期几,第3-0位表示第几节 \\
        \midrule

        \multicolumn{5}{c}{\textbf{数据组9:教学班信息}} \\
        \midrule
        教学班编号 & 某一个课程在某个学期,由某个老师开课的教学班编号 & 整数(32) & 否 & 教学班编号不重复,同一课程老师不同或学期不同,教学班编号均不同 \\
        课程编号 & 教学班的课程编号 & 字符串(24) & 否 & 必须为某个存在的课程的课程编号 \\
        教师工号 & 教学班的教师工号 & 字符串(32) & 否 & 必须为某个存在的教师的工号 \\
        对内容量 & 该教学班针对的学生的容量 & 整数(16) & 否 & 必须大于0。只有课程种类为必修课、选修课、体育课时有对内容量 \\
        对外容量 & 该教学班针对的学生以外的容量 & 整数(16) & 否 & 必须大于0。只有课程种类不为体育课时有对外容量,通识课、科研科面向全体学生开放,只使用对外容量 \\
        额外说明 & 额外说明,如科研课的课题介绍 & 字符串(512) & 是 & 无 \\
        开设学期 & 教学班开设的学期 & 字符串(32) & 否 & 有统一的格式 \\
        使用场地资源 & 该教学班使用的场地资源编号 & 整数(32)集合 & 是 & 必须为某个存在的场地资源的场地资源编号。在分配场地资源之前为空。使用场地资源的数量要与课程的学时一致。任意两个开设学期相同的教学班使用场地资源不得有交集 \\
        \midrule

        \multicolumn{5}{c}{\textbf{数据组10:选课信息}} \\
        \midrule
        学生学号 & 选课的学生的学号 & 字符串(32) & 否 & 必须为某个存在的学生的学号 \\
        课程编号 & 选择课程的课程编号 & 字符串(24) & 否 & 必须为某个存在的课程的课程编号 \\
        教学班编号 & 选择课程的教学班编号 & 整数(32) & 否 & 必须为某个存在的教学班的教学班编号,该教学班的课程编号要与本信息的课程编号相同 \\
        成绩 & 学生选择该课程获得的成绩 & 浮点数(32) & 是 & 0.0-100.0之间的0.5的倍数。在期末考试结束之前没有成绩 \\
        评教分数 & 学生对该课程的评教分数 & 浮点数(32) & 是 & 0.0-100.0之间的0.5的倍数。在评教完成之前没有评教分数 \\
        \midrule

        \multicolumn{5}{c}{\textbf{数据组11:预选信息}} \\
        \midrule
        学生学号 & 进行预选的学生的学号 & 字符串(32) & 否 & 必须为某个存在的学生的学号 \\
        课程编号 & 预选课程的课程编号 & 字符串(24) & 否 & 必须为某个存在的课程的课程编号 \\
        教学班编号 & 预选课程的教学班编号 & 整数(32) & 否 & 必须为某个存在的教学班的教学班编号,该教学班的课程编号要与本信息的课程编号相同 \\
        志愿顺序 & 该课程在学生志愿中的顺序 & 整数(8) & 否 & 对于必修课和选修课,应为0-2;对于通识课和体育课,应为0-4;对于科研课,应为0。并且有等于该值的同课程编号(必修课和选修课)或同课程分类(通识课和体育课)的预选信息条数。\\
        自我介绍 & 学生预选科研课的自我介绍,用于教师选择学生 & 字符串(512) & 是 & 若预选课程是科研课,则不得为空。若预选课程不是科研课,则必须为空。 \\
    \end{longtable}
    \textit{注:类型(宽度)中,字符串括号中数字指字符的数量,整数、枚举和浮点数括号中数字指比特数。}
\end{center}
