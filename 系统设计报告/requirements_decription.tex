\subsection{需求描述}

\subsubsection{系统功能}
\ttfamily
该系统的用户通过{\sffamily 网页}访问系统的各项功能。该系统具有{\sffamily 统一登录}功能,不同用户使用同一入口登录系统。系统自动根据用户的角色展示相应的内容并提供对应的服务。
该系统服务于教学管理的多个环节,包括教师开课、预选抽签、补退改选、成绩公布、教学评价。

\normalfont

\subsubsection{学生}
\begin{enumerate}
    \item \textit{查看课程信息} \quad 在选课开放期间,学生可以查看所有可选课程的信息。课程信息\textbf{分类显示},共分为必修课、选修课、通识课、体育课、科研课五类。可以查看的课程信息包括课程编号、课程名称、学分、上课时间、上课地点、教师姓名、教学班容量、额外说明。可以通过\textbf{搜索}的方式找到需要的课程。
    \item \textit{选课与退课} \quad 在\textbf{预选时间段}内,学生可以在所有可选课程中预选课程。对于必修课和选修课,若有多个由不同老师开设的同一课程的教学班,学生可以选择三个老师开设的教学班并且排出\textbf{志愿顺序};对于通识课和体育课,学生可以在所有同分类的课程中任选五门课程并且排出\textbf{志愿顺序},多个由不同老师开设的同一课程的教学班分别占用志愿;对于科研课,学生可以在所有可选课题中选择一项并提交自我介绍。在\textbf{退改时间段}内,学生可以退选课程,改选同一门课程其他老师的教学班,增选其他课程。对于两个编号不同的教学班,若上课时间存在相同时段,不能同时选中。对于一门课程,只能选中一个老师开设的教学班。学生可以查看自己选退改的\textbf{操作记录}。
    \item \textit{查看课表} \quad 在\textbf{预选时间段}内,学生可以查看自己的预选课表。在\textbf{选课结果公布后},学生可以查看自己的实际课表。课表显示的课程信息包括课程编号、课程名称、上课时间、上课地点、教师姓名。
    \item \textit{查看成绩} \quad 在期末成绩公布之后,学生可以查看自己的成绩。可以查看的成绩信息包括课程编号、课程名称、学分、成绩。学生可以查看过往学期的各门课程成绩以及算数平均分、加权平均分、GPA。
    \item \textit{评价课程} \quad 在\textbf{期末考试后},学生可以对自己所上的课程进行评价。评价的内容包括教学内容、教学方法、教学效果等方面,对于不同种类的课程,评价指标各有侧重,结果以0至100分的评教分数呈现。
    \item \textit{管理个人信息} \quad 学生可以在系统中维护自己的个人信息,主要为登录密码、联系方式。若姓名、性别、院系、专业、班级信息有错误,可以向\textbf{学院教务}申请修改。
\end{enumerate}

\subsubsection{教师}
\begin{enumerate}
    \item \textit{开设课程与修改课程信息} \quad 在\textbf{预选时间段}前,教师可以在系统中开课。开课时从课程清单中选择课程,确定教学班容量(包括对内和对外的容量)。可以附带额外说明,如科研课的课题介绍。若要增开未在清单中的课程或修改由自己申报的课程,需向\textbf{学校管理员}申报,提交课程分类、课程名称、学分、学时、教学大纲等文件,并等待审核。
    \item \textit{查看选课结果} \quad 在\textbf{预选结束后},教师可以查看预选了自己开设的\textbf{科研课}的学生的自我介绍,并且确定最终参与的学生。在\textbf{选课结果公布后},教师可以查看自己所开设课程的教学班的选课结果。选课结果的信息包括学生学号、学生姓名、学生院系、学生班级、学生联系方式。在\textbf{退改时间段}内,选课结果可能变化,教师可随时查看最新结果。若选课人数过少,教师将收到停开提示。同时,系统支持\textbf{导出选课结果为Excel文件}。
    \item \textit{录入成绩与成绩分析} \quad 在\textbf{期末考试后},教师可以录入自己所开设课程的教学班的学生的成绩。可以在系统界面上输入成绩,也可以\textbf{导入指定格式的Excel文件}。录入成绩后,系统会自动计算教学班的平均分、最高分、最低分、及格率等统计信息。可以查看分数分布\textbf{统计图表}。若教师发现成绩录入错误,可以修改成绩。若教师为某一班级的\textbf{班主任}可以查看该班级所有学生的成绩情况。
    \item \textit{管理个人信息} \quad 教师可以在系统中维护自己的个人信息,主要为登录密码、联系方式。若姓名、性别、院系信息有错误,可以向\textbf{学院教务}申请修改。
\end{enumerate}

\subsubsection{学院教务}
\begin{enumerate}
    \item \textit{管理学生和教师} \quad 学院教务可以管理学生的信息,包括学生的姓名、性别、班级。可以管理教师的信息,包括教师的姓名、性别。
    \item \textit{管理班级} \quad 学院教务可以管理班级的信息,包括班级所在的专业、班级的班主任。
    \item \textit{管理选课} \quad 在\textbf{预选时间段}前,学院教务可以\textbf{审核}教师的开课信息并分配场地资源。学院教务可以为每个班级设定可以选择的课程的范围。在\textbf{选课结果公布前},由学院教务对选课结果进行审核,对于选课结果有问题的学生,学院教务可以进行调整。在\textbf{退改时间段},若学生遇到必修课教学班容量不足等问题,学院教务可以调整本学院老师的教学班的容量。
    \item \textit{管理与分析成绩} \quad 学院教务可以统计分析学院所有学生成绩情况和排名情况。可以查看分数分布\textbf{统计图表}。可以按照专业、班级\textbf{进行不同精度的分析}。
    \item \textit{管理教学评价信息} \quad 学院教务可以查看学生对课程的评价情况。可以查看评教分数分布\textbf{统计图表},并向教师反馈。
    \item \textit{管理个人信息} \quad 学院教务可以在系统中维护自己的个人信息,主要为登录密码、联系方式。若姓名、性别、院系信息有错误,可以向\textbf{学校管理员}申请修改。
\end{enumerate}

\subsubsection{学校管理员}
\begin{enumerate}
    \item \textit{管理课程} \quad 学校管理员可以维护课程清单。可以\textbf{审核}教师的增开课程申请,为新的课程指定课程编号,可以\textbf{审核}教师的修改课程申请,可以从清单中删除不再开设的课程。
    \item \textit{管理选课} \quad 学校管理员负责进行学期的设置和\textbf{选课阶段的设置}。学校管理员负责启动预选的\textbf{抽签}。抽签的原则是每个课程的各教学班按照志愿顺序从高到低抽取,同一课程所有的教学班抽完了某一志愿再继续抽取下一志愿。先从第一志愿预选该教学班的学生当中抽取(已满视为抽完了该志愿)。若未抽满,则依次从后续志愿预选该教学班的学生中抽取,直到抽满或者到达教学班允许的最大志愿数。对于必修课或选修课,学生若在某一志愿被对应的教学班抽中,不参与同一课程后续志愿的教学班抽签。学校管理员可以查看抽签结果。
    \item \textit{管理院系} \quad 学校管理员可以进行院系信息的管理。可以创建新的院系,管理院系的名称和下设的专业。
    \item \textit{管理场地资源} \quad 学校管理员可以进行场地资源的管理。可以增加场地、修改场地信息、设置场地是否可信。
    \item \textit{管理个人信息} \quad 学校管理员可以查看和管理所有用户的个人信息。可以\textbf{批量地添加、修改、删除}人员信息。
    \item \textit{管理系统运行信息} \quad 学校管理员可以查看系统的运行情况,包括系统当前的运行信息、统计数据、\textbf{各操作的审计信息}等。
\end{enumerate}
