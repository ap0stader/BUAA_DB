\subsection{需求描述}

\subsubsection{系统功能}
\ttfamily
该系统的用户通过{\sffamily 网页}访问系统的各项功能。该系统具有{\sffamily 统一登录}功能,不同用户使用同一入口登录系统。系统自动根据用户的角色展示相应的内容并提供对应的服务。
该系统服务于教学管理的多个环节,包括教师开课、预选抽签、补退改选、成绩公布、教学评价等。

\normalfont

\subsubsection{学生}
\begin{enumerate}
    \item \textit{查看课程信息} \quad 在选课开放期间,学生可以查看所有可选课程的信息。课程信息\textbf{分类显示},共分为必修课、选修课、通识课、体育课、科研课五类。可以查看的课程信息包括课程编号、课程名称、学分、上课时间、上课地点、教师姓名、课程容量、额外说明。可以通过\textbf{搜索}的方式找到需要的课程。
    \item \textit{选课与退课} \quad 在\textbf{预选时间段}内,学生可以在所有可选课程中预选课程。对于必修课和选修课,若多个老师开设编号相同的课程,学生可以选择三个老师开设的课程并且排出\textbf{志愿顺序};对于通识课和体育课,学生可以在所有同分类的课程中任选五门;对于科研课,学生可以在所有课题中选择一项并提交自我介绍。在\textbf{退改时间段}内,学生可以退选课程,改选同一门课程的其他老师,增选其他课程。对于两门课程编号不同的课程,若上课时间存在相同时段,不能同时选择。学生可以查看自己选退改的操作记录。
    \item \textit{查看课表} \quad 在\textbf{预选时间段}内,学生可以查看自己的预选课表。在\textbf{选课结果公布后},学生可以查看自己的实际课表。课表显示的课程信息包括课程编号、课程名称、上课时间、上课地点、教师姓名。
    \item \textit{查看成绩} \quad 在期末成绩公布之后,学生可以查看自己的成绩。可以查看的成绩信息包括课程编号、课程名称、学分、成绩。学生可以查看过往学期的各门课程成绩以及个人的算数平均分、加权平均分、GPA。
    \item \textit{评价教师} \quad 在期末考试结束之后,学生可以对自己所上的课程的教师进行评价。评价的内容包括教学内容、教学方法、教学效果等方面,对于不同种类的课程,评价指标各有侧重,结果以0-100分的分数呈现。
    \item \textit{管理个人信息} \quad 学生可以在系统中维护自己的个人信息,主要为登录密码、联系方式。若姓名、性别、院系、专业、班级信息有错误,可以向\textbf{学院教务}申请修改。
\end{enumerate}

\subsubsection{教师}
\begin{enumerate}
    \item \textit{开设课程与修改课程信息} \quad 在\textbf{预选时间段}前,教师可以在系统中开设课程。开设课程时从课程清单中选择开设课程的课程编号,确定课程容量(包括对内和对外的容量)、上课时间,可以附带额外说明。若要增开未在清单中的课程或修改由自己申报的课程,需向\textbf{学校管理员}申报,提交课程分类、名称、学分、教学大纲等文件,并等待审核。
    \item \textit{查看选课结果} \quad 在\textbf{预选结束后},教师可以查看预选了自己开设的\textbf{科研课}的学生的自我介绍,并且确定最终参与的学生。在\textbf{选课结果公布后},教师可以查看自己所开设课程的选课结果。选课结果的信息包括学生学号、学生姓名、学生院系、学生班级、学生联系方式。在\textbf{退改时间段}内,选课结果可能变化,教师可随时查看最新结果。若选课人数过少,教师将收到该课程停开的提示。同时,系统支持教师\textbf{导出选课结果为Excel文件}。
    \item \textit{录入成绩与成绩分析} \quad 在\textbf{期末考试后},教师可以录入自己所开设课程的学生的成绩。可以在系统界面上输入成绩,也可以\textbf{导入指定格式的Excel文件}。录入成绩后,系统会自动计算平均分、最高分、最低分、及格率、分数分布等统计信息,教师可以查看这些\textbf{统计图表}。若教师发现成绩录入错误,可以修改成绩。若教师为某一班级的\textbf{班主任}可以查看该班级各学生的成绩情况。
    \item \textit{管理个人信息} \quad 教师可以在系统中维护自己的个人信息,主要为登录密码、联系方式。若姓名、性别、院系信息有错误,可以向\textbf{学院教务}申请修改。
\end{enumerate}

\subsubsection{学院教务}
\begin{enumerate}
    \item \textit{管理学生和教师} \quad 学院教务可以管理学生的信息,包括学生的姓名、性别、学号、班级。院教务可以管理教师的信息,包括教师的姓名、性别、工号。
    \item \textit{管理班级} \quad 学院教务可以管理班级的信息,包括班级所在的专业、班级的班主任。
    \item \textit{管理选课} \quad 在\textbf{预选时间段}前,学院教务可以查看并管理教师的开设课程信息,根据教师提供的上课时间分配教室。学院教务可以为每个班级设定可以选择的课程的范围。在\textbf{选课结果公布前},由学院教务对选课结果进行审核,对于选课结果有问题的学生,学院教务可以进行调整。在\textbf{退改时间段},若学生遇到必修课课程容量不足等问题,学院教务可以调整由本学院老师开设的课程容量。
    \item \textit{管理与分析成绩} \quad 学院教务可以统计分析学院所有学生成绩情况,查看包括分数分布、排名情况等\textbf{统计图表},可以按照专业、班级等\textbf{不同精度进行分析}。
    \item \textit{管理教学评价信息} \quad 学院教务可以查看学生对教师的评价情况,可以查看评教分数分布等\textbf{统计图表},并向教师反馈。
    \item \textit{管理个人信息} \quad 学院可以在系统中维护自己的个人信息,主要为登录密码、联系方式。若姓名、性别、工号、院系信息有错误,可以向\textbf{学校管理员}申请修改。
\end{enumerate}

\subsubsection{学校管理员}
\begin{enumerate}
    \item \textit{管理课程} \quad 学校管理员可以维护课程清单。可以\textbf{审核}教师申报的增开课程申请,为新的课程指定课程编号,从清单中删除不再开设的课程。
    \item \textit{管理选课抽签} \quad 学校管理员负责启动预选的\textbf{抽签}。抽签的原则是每个课程按照学生预选的志愿顺序从高到低抽取,先从第一志愿预选该课程的学生当中抽取。若未抽满,则依次从后续志愿预选该课程的学生中抽取,直到抽满或者到达该类课程允许的最大志愿数。学校管理员可以查看抽签结果。
    \item \textit{管理院系} \quad 学校管理员可以进行院系信息的管理。可以创建新的院系,管理院系的名称和下设的专业。
    \item \textit{管理个人信息} \quad 学校管理员可以查看和管理所有用户的个人信息。可以\textbf{批量地添加、修改、删除}人员信息。
    \item \textit{管理系统运行信息} \quad 学校管理员可以查看系统的运行情况,包括系统当前的运行信息、统计数据、\textbf{各操作的审计信息}等。
\end{enumerate}
