\subsection{需求描述}

\subsubsection{系统功能}
\ttfamily
该的用户通过浏览器{\sffamily 网页}访问系统的各项功能。该系统具有{\sffamily 统一登录}功能,不同用户使用同一入口登录系统。系统自动根据该用户的角色及其所拥有的权限展示相应的内容并提供对应的服务。
该系统服务与本科教学管理的多个环节包括教师开课、预选抽签、补退改选、期末考试、成绩公布、教学评价等。

\normalfont

\subsubsection{学生}
\begin{enumerate}
    \item \textit{查看课程信息} \quad 在选课开放期间,学生可以查看所有可选课程的信息。课程信息\textbf{分类}显示。课程共分为必修课、选修课、通识课、体育课、科研课五类。查看的信息包括课程编号、课程名称、学分、上课时间、上课地点、教师、课程容量。
    \item \textit{选课与退课} \quad 在\textbf{预选时间段}内,学生可以在所有可选课程中预选课程。对于有多个老师开设的课程编号相同的课程,学生可以选择三个老师并且排出志愿顺序。对于两门课程编号不同的课程,若上课时间相同,不能同时选择。对于通识课和体育课,学生可以在所有该类别的课程中任选五门,对于科研课,学生可在所有课题中选择一项并提交自我介绍。在\textbf{退改时间段}内,学生可以退选课程,改选同一门课程的其他老师,增选其他课程。同一时间段内,学生只能上一门课程。
    \item \textit{查看课表} \quad 在选课结果公布之后,学生可以查看自己的课表。课表显示的信息包括课程编号、课程名称、上课时间、上课地点、教师。
    \item \textit{查看成绩} \quad 在期末成绩公布之后,学生可以查看自己的成绩。成绩显示的信息包括课程编号、课程名称、学分、成绩。
    \item \textit{评价教师} \quad 在期末考试结束之后,学生可以对自己所上的课程的教师进行评价。评价的内容包括教学内容、教学方法、教学效果等方面,对于不同种类的课程,评价指标各有侧重。
    \item \textit{管理个人信息} \quad 学生可以在系统中维护自己的个人信息,主要为登录密码、联系方式。若姓名、性别、学号、院系、专业、班级等信息有错误,可以向学院教务申请修改。
\end{enumerate}

\subsubsection{教师}
\begin{enumerate}
    \item \textit{开设课程与修改课程信息} \quad 
    \item \textit{查看选课结果} \quad 
    \item \textit{查看上课课表} \quad 
    \item \textit{录入成绩与成绩分析} \quad 
    \item \textit{管理个人信息} \quad 
\end{enumerate}

\subsubsection{学院教务}
\begin{enumerate}
    \item \textit{管理学生} \quad 
    \item \textit{管理教师} \quad 
    \item \textit{管理课程} \quad 
    \item \textit{管理选课} \quad 
    \item \textit{管理与分析成绩} \quad 
    \item \textit{管理教学评价信息} \quad 
    \item \textit{管理个人信息} \quad 
\end{enumerate}

\subsubsection{学校管理员}
\begin{enumerate}
    \item \textit{学院教务各权限} \quad 
    \item \textit{管理院系信息} \quad 
    \item \textit{管理班级信息} \quad 
    \item \textit{管理系统运行信息} \quad 
\end{enumerate}
