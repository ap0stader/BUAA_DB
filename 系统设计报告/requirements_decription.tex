\subsection{需求描述}

\subsubsection{系统功能}
\ttfamily
该的用户通过浏览器{\sffamily 网页}访问系统的各项功能。该系统具有{\sffamily 统一登录}功能,不同用户使用同一入口登录系统。系统自动根据该用户的角色及其所拥有的权限展示相应的内容并提供对应的服务。
该系统服务与本科教学管理的多个环节包括教师开课、预选抽签、补退改选、期末考试、成绩公布、教学评价等。

\normalfont

\subsubsection{学生}
\begin{enumerate}
    \item \textit{查看课程信息} \quad 在选课开放期间,学生可以查看所有可选课程的信息。课程信息\textbf{分类显示}。课程共分为必修课、选修课、通识课、体育课、科研课五类。查看的信息包括课程编号、课程名称、学分、上课时间、上课地点、教师、课程容量。
    \item \textit{选课与退课} \quad 在\textbf{预选时间段}内,学生可以在所有可选课程中预选课程。对于有多个老师开设的课程编号相同的课程,学生可以选择三个老师并且排出志愿顺序。对于两门课程编号不同的课程,若上课时间相同,不能同时选择。对于通识课和体育课,学生可以在所有该类别的课程中任选五门,对于科研课,学生可在所有课题中选择一项并提交自我介绍。在\textbf{退改时间段}内,学生可以退选课程,改选同一门课程的其他老师,增选其他课程。同一时间段内,学生只能上一门课程。
    \item \textit{查看课表} \quad 在预选时间段内,学生可以查看自己的预选课表。在选课结果公布之后,学生可以查看自己的实际课表。课表显示的信息包括课程编号、课程名称、上课时间、上课地点、教师。
    \item \textit{查看成绩} \quad 在期末成绩公布之后,学生可以查看自己的成绩。成绩显示的信息包括课程编号、课程名称、学分、成绩。同时可以查看过往学期的成绩以及个人的平均分、GPA和排名信息。
    \item \textit{评价教师} \quad 在期末考试结束之后,学生可以对自己所上的课程的教师进行评价。评价的内容包括教学内容、教学方法、教学效果等方面,对于不同种类的课程,评价指标各有侧重。
    \item \textit{管理个人信息} \quad 学生可以在系统中维护自己的个人信息,主要为登录密码、联系方式。若姓名、性别、学号、院系、专业、班级信息有错误,可以向学院教务申请修改。
\end{enumerate}

\subsubsection{教师}
\begin{enumerate}
    \item \textit{开设课程与修改课程信息} \quad 在选课开放之前,教师可以在系统中开设课程。开设课程时从课程清单中选择开设的课程分类、课程编号,确定课程容量(包括对内和对外的容量)、上课时间。若要增开未在清单中的课程或修改由自己申报的课程,需向学校管理员申报,提交课程分类、名称、学分、教学大纲文件等并等待审核通知。
    \item \textit{查看选课结果} \quad 在选课结果公布之后,教师可以查看自己所开设课程的选课结果。选课结果显示的信息包括学生学号、学生姓名、学生班级、学生联系方式。退改时间段内,选课结果可能变化,教师可随时查看最新结果。若选课人数过少,教师将收到该课程停开的通知。同时,系统支持教师\textbf{导出选课结果为Excel文件}。
    \item \textit{录入成绩与成绩分析} \quad 在期末考试结束之后,教师可以录入所教课程的成绩。可以在系统界面上输入成绩,也可以\textbf{导入给定格式的Excel文件}。录入成绩后,系统会自动计算平均分、最高分、最低分、及格率、分数分布等统计信息。教师可以查看这些\textbf{统计信息图表}。若教师发现成绩录入错误,可以修改成绩。若教师为一班级的班主任可以查看该班级各学生的成绩情况。
    \item \textit{管理个人信息} \quad 教师可以在系统中维护自己的个人信息,主要为登录密码、联系方式。若姓名、性别、工号、院系信息有错误,可以向学院教务申请修改。
\end{enumerate}

\subsubsection{学院教务}
\begin{enumerate}
    \item \textit{管理学生} \quad 学院教务可以管理学生的信息,包括学生的姓名、性别、学号、专业、班级。学院教务可以查看并管理学生的选课信息。查看成绩情况、评教情况。对于从其他学院转入的学生,学院教务审核其转入申请并将其安排至一个班级中。
    \item \textit{管理教师} \quad 学院教务可以管理教师的信息,包括教师的姓名、性别、工号。
    \item \textit{管理班级} \quad 学院教务可以管理班级的信息,包括班级所在的专业、班级的班主任。
    \item \textit{管理选课} \quad 在选课开放之前,学院教务可以查看并管理教师的开设课程信息,根据教师提供的上课时间分配教室。为每个班级设定设定可以选择的课程的范围。在选课结果公布之前,由学院教务对选课结果进行审核,对于选课结果有问题的学生,学院教务可以进行调整。在退改选阶段,若学生遇到必修课课程容量不足等问题,学院教务可以调整由本学院老师开设的课程容量。
    \item \textit{管理与分析成绩} \quad 学院教务可以统计分析学院所有学生成绩情况,包括分数分布、排名情况等统计图表。
    \item \textit{管理教学评价信息} \quad 学院教务可以查看学生对教师的评价情况,包括评价内容、评价分数分布等,并向教师反馈。
    \item \textit{管理个人信息} \quad 学院可以在系统中维护自己的个人信息,主要为登录密码、联系方式。若姓名、性别、工号、院系信息有错误,可以向学校管理员申请修改。
\end{enumerate}

\subsubsection{学校管理员}
\begin{enumerate}
    \item \textit{学院教务各权限} \quad 学校管理员拥有各学院教务的所有权限(除审核转入申请)。
    \item \textit{管理院系} \quad 学校管理员可以进行院系信息的管理。可以创建新的院系、管理院系的编号、名称。可以管理院系的学生、教师、教务人员信息。批量地添加、修改、删除人员信息。对于学生在院系之间的转移,学校管理员向转入学院转交学生的转入申请。
    \item \textit{管理课程} \quad 学校管理员维护课程清单。可以审核教师申报的增开课程申请,为新的课程指定课程编号。从清单中删除不再开设的课程。
    \item \textit{管理个人信息} \quad 学校管理员可以在系统中维护自己的个人信息,主要为登录密码。同时可以查看和管理所有用户的个人信息。
    \item \textit{管理系统运行信息} \quad 学校管理员可以查看系统的运行情况,包括系统当前的运行信息、统计数据、\textbf{各操作的审计信息}等。
\end{enumerate}
